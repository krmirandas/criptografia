\documentclass[11pt]{article} % No modificar
\usepackage[margin=0.5cm, bottom=1.5cm, a5paper]{geometry} % No modificar
\usepackage[utf8]{inputenc}
\usepackage[spanish]{babel}
\usepackage{amsmath}
\usepackage{amsfonts}
\usepackage{amssymb}
\usepackage{graphicx}
\usepackage{hyperref}
\usepackage{enumitem}

\newcommand{\ZZ}{\mathbb{Z}}

\title{Tarea 1}
\author{Un Autor \and Otro autor}
\date{}

\begin{document}
	\maketitle

	\subsection*{Respuestas}
	\begin{enumerate}
		\item Descifra los siguientes mensajes que fueron cifrados con el método de César, probando diferentes desplazamientos hasta que el mensaje tenga sentido. Escribe el mensaje claro y la llave (desplazamiento) que se usó para cifrar.
		\begin{enumerate}
			\item \texttt{SLYDPYQCGLQNGPYBMPY}
			\item \texttt{CVVCEMVJGKORNGOGPVCVKQP}
			\item El archivo \texttt{imagen.enc} que originalmente era una imagen.
		\end{enumerate}

		\item Considera la siguiente tabla de cifrado de sustitución simple
		\begin{table}[ht]
			\centering
			\resizebox{0.9\textwidth}{!}{
			\begin{tabular}{|cccccccccccccccccccccccccc|}
				\hline 
			A & B & C & D & E & F & G & H & I & J & K & L & M & N & O & P & Q & R & S & T & U & V & W & X & Y & Z \\
				\hline 
			W & P & U & B & A & Q & O & Y & G & C & Z & E & F & M & J & V & D & K & I & R & H & L & T & S & N & X \\
				\hline 
			\end{tabular}}
		\end{table}
		\begin{enumerate}
			\item Encripta el mensaje
			\begin{center}
				\texttt{Criptografia y seguridad}.
			\end{center}
			\item Escribe la tabla correspondiente que se usa para descifrar, la primera fila debe ser el alfabeto en orden.
			\item Usando tu tabla del inciso anterior, descifra el mensaje
			\begin{center}
				\texttt{RGFGMOWRRWUZIWKAWIGOMGQGUWMRRYKAWRRJUKNVRJGFVEAFAMRWRGJMI}
			\end{center}
			
			\item ¿Cómo sería una tabla de cifrado si los mensajes fueran cadenas de bytes (archivos) en vez de las 26 letras del alfabeto? ¿De qué tamaño sería la tabla?
		\end{enumerate}
	
		\item (2 pts.) El texto del archivo \texttt{texto.enc} fue cifrado con el método de sustitución simple. El original es un texto en español, encuéntralo.
		
		\item En cada inciso encuentra el valor de $ x $ entre $ 0 $ y $ m-1 $ que resuelve la congruencia, donde $ m $ es el módulo.
		\begin{enumerate}
			\item $ 123 + 513 \equiv x \pmod{763} $.
			\item $ 222^3 \equiv x \pmod{581} $.
			\item $ x - 21 \equiv 23 \pmod{37} $.
			\item $ x^{2} \equiv 5 \pmod{11} $.
			\item $ x^{3} - 2x^{2} + x - 2 \equiv 0 \pmod{11} $.
		\end{enumerate}
	
		\item Sea $ m \in \ZZ $.
		\begin{enumerate}
			\item Supón que $ m $ es impar. Encuentra el entero entre $ 1 $ y $ m-1 $ que es igual a $ 2^{-1} \pmod{m} $.
			\item De forma más general, supón que $ m \equiv 1 \pmod{b} $. Encuentra el entero entre $ 1 $ y $ m-1 $ que es igual a $ b^{-1} \pmod{m} $,
		\end{enumerate}
		
		\item Explica por qué las siguientes funciones no sirven para encriptar mensajes considerando que los espacios de mensajes y llaves son iguales a $ \ZZ/N = \{0,1,\ldots,N-1\} $.
		\begin{enumerate}
			\item $ E(k,m) = km \pmod{N} $.
			\item $ E(k,m) = (k+m)^{2} \pmod{N} $.
		\end{enumerate}
		
		\item (2 pts.) Considera el cifrado afín con una llave $ k = (k_1, k_2) $.
		\begin{enumerate}
			\item Usando $ N = 101 $ y $ k=(99,20) $, cifra el mensaje $ m = 100 $ y descifra el criptotexto $ c = 23 $.
			\item Describe un ataque de texto claro conocido para recuperar la llave $ (k_1, k_2) $. Observa que la función de cifrado es la ecuación de una recta en el plano, donde las coordenadas corresponden a una letra en claro y una letra cifrada, ¿cuántos puntos de una recta se necesitan para determinar su ecuación?
			\item Aplica tu ataque al archivo cifrado \texttt{audio.enc}, que originalmente es un audio en formato MP3. Es posible que tengas que modificar un poco el ataque.
		\end{enumerate}
	
		\item Muestra que los esquemas de César, sustitución simple y Vigenère pueden romperse fácilmente con un ataque de texto claro elegido. ¿Cuántos mensajes claros se necesitan para recuperar la llave en cada caso?
	\end{enumerate}

\end{document}